%2014 9 
%用到了 \usepackage{CJKutf8) 作为中文输入
%用到了图片并排的格式
%用到了 ~~  \  \  作为空格
%
\documentclass{article}
\usepackage{CJKutf8}
\usepackage{graphicx}%tu pian pai ban
\usepackage{subfigure}

\title{实验班科研及境外交流总结}
\author{朱彬\\逸仙学院~~理工院~~12级~~物理学}
\date{2014年 9月}

\begin{document}


\begin{CJK*}{UTF8}{gbsn}
\maketitle
\section{科研情况}

\ \ \ \ 进入光电材料与技术国家重点实验室陈军教授研究组开展科研工作,主要研究工作为纳米线场发射的特性和机理研究。

在纳米线场发射方面,我在陈军教授的指导下,阅读和研究文献并熟悉实验室测量场发射的实验仪器。开展了温度对少层石墨烯场发射特性的研究。下面几张图是一些实验过程的照片和一些实验结果的展示。
\begin{figure}[h!]
\begin{minipage}[t]{0.5\linewidth}
\centering
\includegraphics[scale=0.03]{sample.jpg}
\caption{实验样本}
\label{fig:sample}
\end{minipage}%
\begin{minipage}[t]{0.5\linewidth}
\centering
\includegraphics[scale=0.17]{result.jpg}
\caption{温度对少层石墨烯场发射特性的影响}
\label{fig:result}
\end{minipage}%
\end{figure}

\section{获奖情况}
2013年度中山大学优秀团员\\
2012-2013学年度中山大学优秀学生一等奖学金\\
2012-2013学年度许崇清奖学金

\section{境外交流}
~~~~~2014年暑期赴欧洲核子中心学习和实习。

我们六个人很幸运可以来到欧洲核子中心(CERN)参加暑期学校的学习。我和林睿,李红喜同学在CERN的暑期课程结束后,加入AMS02的台湾组,进行科研训练。这个部分将分两个部分介绍这次活动的我的感悟:一是在CERN参加暑期课程时的感悟,二是在AMS参与科研训练时的感悟。
\subsection{关于在CERN的summer lectures 的学习情况}
~~~~暑期课程总共是六周的时间,从7月1号开始。在此期间,周一到周五每天早上有三节课,下午的时间里会有一些讲座,研讨会(workshop),或者参观探测器,实验室等的活动可以参加。
\subsubsection{课程感悟}
~~~~CERN的暑期课程内容很多,每一门课一般是三个课时左右,关于理论方面的课程一般会有四个课时,实验方面有一些课程是两个课时。课程内容基本覆盖了粒子物理的理论以及CERN实验中各个子系统,如探测器,粒子加速器等等相关的内容。关于课程的感悟,我和其他同伴比较相似的地方是我们在理论课上能听懂的内容较多,在介绍实验,特别是介绍硬件方面的内容上我听懂的不多。这些内容丰富的课程很大程度地拓宽了我们的科研视野。

因为我们住的离CERN较远,所以每天的课程是我们接触其他暑期学生,并与他们进行交流的主要机会。这里有来自世界各地的暑期学生,他们有的多才多艺(有三位暑期学生创作了一首关于弦论的歌曲并在课件表演),他们有的经常提很有质量的问题,和老师同学们进行讨论,他们无条件的容忍我们的Chinglish。我们交到了以为来自丹麦的朋友,感受到他的热情与包容。
\subsubsection{参与研讨会,参观探测器的感悟}
~~~~CERN为暑期学生提供了众多内容丰富而有趣的研讨会(workshop),我参加了关于ROOT学习的workshop,了解了ROOT在高能粒子物理的实验数据分析中强大功能,在讨论和实践中学习和入门ROOT。另外,我还参加了关于自己搭建一个可以探测来自宇宙射线的粒子轨迹的云室的workshop,自己动手制作了一个云室,并观察粒子轨迹。实验过程和实验后与其他同学之间的讨论都十分有趣。

八月初,我有机会参观LHC中最大规模的探测器ATLAS。了解了ATLAS一些信息,深深被探测器的大规模和精细到极致的每部分结构所震撼。
begin{figure}[h!]
\centering
\includegraphics[scale=0.05]{visit.jpg}
\caption{参观ATLAS实验探测器}
\label{fig:visit}
\end{figure}

\subsection{关于在AMS02学习和科研的情况}
~~~~暑期课程主要是在每天上午,所以前六周的每天下午如果没有参加研讨会或者参观等活动,我们会来到CERN2的AMS的办公室,为科研训练做一些前期的准备。在此期间,我主要学习了CERN服务器的一些操作,即学习Linux系统的主要命令行操作,以及文本编辑器vi的基本操作,入门要用的统计分析框架ROOT。此外,在博士后Julia 和组里的李慧玲师姐的知道下学习AMS探测器中各个子系统的功能及相关的知识信息。

由于关于计算机的背景知识比较少,上手Linux系统的时间花的比较多,但这个时间花的很值,在学习Linux系统的基本命令以及简单的bash Shell编程的过程中我比之间更加深入了解了一些计算机方面的问题,个人觉得是对我挺有帮助的。学习文本编辑器vi使得我可以比较高效地编写处理实验要用的代码。
ROOT的学习过程也是比较艰辛,因为它的功能发展到如今是十分强大,以至于要学习的东西很多,对刚入门的用户来说会比较难上手,我大概花了一个月左右的时间入门了ROOT,基本可以看懂师姐给我的代码,以及自己编写一些处理数据的代码。

    暑期课程结束后,周一到周五我们便呆在AMS进行科研训练,师姐让我参与到她的课题中,进行关于利用AMS的蒙特卡洛数据及探测粒子束(Test beam)数据测量C的散射截面的实验数据分析。我们现在主要是在计算一些cut的效率,以及选择可以作为测量散射界面的好的数据。和师姐的讨论让我更好的理解AMS探测器是怎样工作的,以及实验数据背后的物理含义。
\begin{figure}[h!]
\centering
\includegraphics[scale=0.05]{ams_office.jpg}
\caption{AMS02实验的主办公室~~~~丁肇中教授的办公室所在地}
\label{fig:ams_office}
\end{figure}
    
    每天下午5点是参加AMS的组会。组会一般由丁肇中教授主持,每天都会有专门的教授报告探测器各个部分的工作情况,是我们更好的了解各个子探测器。经常会有组里的成员进行presentation展示他们的研究情况,就具体问题进行讨论,学术研究的氛围很浓厚。
\subsection{总结}
~~~~加入实验班这一年,在科研和学习各个方面都学到了很多东西。在接下来的学习和科研的中,我希望能够再接再厉,不断提升自己各方面的素质。很感谢逸仙学院和理工院对我们的支持和帮助!
\end{CJK*}


\end{document}